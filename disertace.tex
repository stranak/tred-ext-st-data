%!TEX TS-program = xelatex
%!TEX encoding = UTF-8 Unicode
\documentclass[11pt, a4paper]{article}

\usepackage{fontspec, xunicode, xltxtra}
\defaultfontfeatures{Mapping=tex-text}
\setmainfont{Adobe Caslon Pro}
\setmonofont[Scale=MatchLowercase]{Monaco}
\setmathsf{Cambria Math}

\usepackage{paralist} % for better itemize and enumerate
% asparaenum, inparaenum, \begin{compactenum}[{Example} a)] etc.
% Also, items can be referenced via \label{} and \ref{}
%
\frenchspacing
\usepackage{natbib}
\bibpunct{(}{)}{;}{a}{,}{,}
\usepackage{gb4e}
\usepackage{wrapfig}

% my macros
\def\code{\texttt}
\def\tred{\texttt{TrEd}}
\def\seman{\texttt{SemAnn}}
\def\astst{$\ast$.st}
\def\sdata{s-data}
\def\stn{st-node}
\def\sn{s-node}
\def\tn{t-node}
\def\stf{st-file}
\def\sf{s-file}
\def\tf{t-file}
\def\mwe{MWE}
\def\Mwe{multi-word expression}
\newcommand\Sref[1]{Section~\ref{#1}}

\title{Thesis Notes}
\author{\textsc{Pavel Straňák}}

\begin{document}
\maketitle

%%%%%%%%%%
\section*{Motto}
\begin{tabular}{@{} lp{11cm} @{}} % @{} removes default spaces
Frasier: & ``How was your hunting trip?''\\
Martin: & ``Came home empty handed.''\\
Frasier: & ``Oh dear; I guess that means for the next several weeks we'll hear your grouse about the grouse and carp about the carp.''\\
Niles: & ``You've been working on that, haven't you?''\\
Frasier: & ``Well, there was traffic.''\\
 & \raggedleft\emph{Frasier, Season 9, Part 3}\\
\end{tabular}

\section*{Idioms}
Even ``non-compositional'' idioms are actually (originaly) metaphorical or methonymical.  Even though sometimes it is hard to see that. At other times a speaker may forget that rather straightforward metaphoric aspect:

\begin{quote}
Barack Obama accused his Republican rivals of stirring a controversy over a comment he made about putting “lipstick on a pig.” \emph{(NY~Times, 11.~September 2008)}
\end{quote}


%%%%%%%%%%%%%%%%%%%%%%%%%%%%%%%%%%%%%%%
\section{PDT 2.0}\label{PDT}
In PDT there are several functors that refer to multiword expressions (MWEs) in one way or another. There are also some technical lemmas (??? nebo jen 1?) like {\tt \#Forn} that identify roots of subtrees representing MWE's.

There are currently two graphical search engines for PDT: Netgraph \cite{netgraph} and TrEd \cite{tred}. Both have their respective benefits, but since TrEd is considerably faster due to its use of an SQL database backend \cite{pmltq}, we have used TrEd for all the examples in this work. We also give the search queries using the PML Tree Query language \cite{pmltq} where appropriate.

%%%%%%%%%%
\subsection{Foreign Phrase: \code{t-node [t\_lemma = \#Forn]}}\label{PDT:Forn}
Foreign Phrase seems to be overused and its overuse seems a bit arbitrary. \\
- jmena firem jsou nekdy forn, nekdy ne. (dohledat)\\

%%
\subsubsection{Foreign phrases with just one t-node}
There are 34 occurrences of this construction in the PDT~2.0. Counting them is as easy as writing a query in Figure~\ref{fig:tq-forn1} and extending it with this filter: \code{>>count(\$n)}.

\begin{wrapfigure}{r}{0.32 \textwidth}
\includegraphics[width=0.3 \textwidth]{images/vyhledavky/query-forn-1-x.pdf}
\caption{PML-TQ search query for single-node foreing phrases}
\label{fig:tq-forn1}
\end{wrapfigure}

In case of the bibliographic reference in Figure~\ref{fig:forn-biblio} there is coordination of three foreign phrases corresponding to the parts of a bibliographic reference annotated, but the reason for this is not very clear. After all, the point of annotating foreign phrases as simple lists with a \code{\#Forn} node as a head was to make no assumptions about these pieces of a text \pageref{pdt-t-man:300}.  \\
- je to kvuli te interpunkci???

\begin{figure}[h]
\includegraphics[width=\textwidth]{images/vyhledavky/forn-coord1x-biblio.pdf}
\caption{A bibliographic reference analysed as a coordination of three foreign phrases}
\label{fig:forn-biblio}
\end{figure}

Names of companies seem to be distinguished more by the country of origin then by any linguistic reasons, as demonstrated in Figure~\ref{fig:forn-firmy}. As far as linguistic criteria are concerned, Chemapol and Inekon are as foreign as Agip or Total. However the first two are, or at least were%
\footnote{At the time of writing this thesis Chemapol is owned by another international company, which only emphasises vagueness of this distinction} %
%
, Czech companies, while those in the latter group have a foreign origin.

\begin{figure}
\includegraphics[width=\textwidth]{images/vyhledavky/nazvy-firem.pdf}
\caption{Annotation of Czech and foreign company names}
\label{fig:forn-firmy}
\end{figure}

%%%%%%%%%%
\subsection{CPHR}
There are 76 occurrences of {\tt CPHR} nodes, whose head verb is not its parent, but only effective parent, in 40 sentences. See Figure~\ref{fig:tq-echild} for the query and Figures~\ref{fig:cphr-echild} and~\ref{fig:cphr-echild2} for examples.
\begin{wrapfigure}{r}{0.32 \textwidth}
\includegraphics[width=0.3 \textwidth]{images/vyhledavky/query-echild.pdf}
\caption{PML-TQ search query for CPHR nodes, whose effective parrent is not its parent}
\label{fig:tq-echild}
\end{wrapfigure}

\begin{figure}[h]
\includegraphics[width=0.8\textwidth]{images/vyhledavky/cphr-echild.pdf}
\caption{Coordination of verbonominal idioms, where the verbal parts are further ???rozvite }
\label{fig:cphr-echild}
\end{figure}

Figure~\ref{fig:cphr-echild2} shows on the other hand a coordination of two V-N idioms with the same verbal part.
\begin{figure}[h]
\includegraphics[width=\textwidth]{images/vyhledavky/cphr-ukladat-sankce-a-penale.pdf}
\caption{Coordination of two idioms starting with the same verb}
\label{fig:cphr-echild2}
\end{figure}

\newpage
\section{Notes}

{\em``Prokopnout pětku''} -- prokopnout [trestný kop after scoring] pětka -- ellipsis crossing into pragmatics(?), because it is {\em not} an ellipsis of some exact linguistic construction, but rather an ellipsis of a situation which everyone in the discourse understand.  

``Návrh / novela Zákona na ochranu osobních údajů''  $$((\mathrm{návrh} \lor \mathrm{novela}) | \mathrm{zákon} \lor \mathrm{předpis})$$
-- lexikální funkce?? \citep{wanner} ?

\subsection{Missing t-nodes in \code{coord}s}
``\textit{První} a druhá světová válka'' apod. -- the word ``První'' is an ellipsis of ``První světová válka''. The ellided t-nodes that should had been newly established were not, however, which left us two options, either annotate a single-word (and single-node) expression as an instance of a multi-word lexeme, or annotate the words ``světová válka'' that occure in the text as being a part of both expressions. We decided for the first, because the agreement on ellipses is fairly high in this type of coordinations.

% !TEX root = ../disertace.tex
\section{\sdata}
\subsection{The design and the PML schema}
\sdata\ means s-layer PML files and the PML schema of these files. The idea behind \sdata\ design is to have a simple way to store additional ``sense'' annotations over any layer of PDT. The annotations are stored as a set of ``sense'' nodes where each s-node contains a link to a sense repository (annotation dictionary) and a set of references to nodes (m-, a- or t-) that correspond to an instance of the sense. An \sf\ is thus basically a very simple flat list of \sn{}s. It does not contain any trees. A single \sf\ can only reference a single PDT file: either tectogrammatical, or analytical, or even morphological layer can be used, but references to different layer cannot be mixed.

\subsection{Visualisation}
There are two basic ways to view st-nodes: in \seman\ or in \tred. Both of these need to use the ``t-a-m-w-'' PDT files to display the sentence and/or the tree for each sentence and then they read the \stf\ to add the information about \stn{}s. The \stn{}s are displayed as colour boxes or bubbles over the words in a sentence or nodes in a tree in \seman\ or \tred\ respectively.

\subsubsection{\seman}
The visualisation of annotated files in \seman\ has the advantage of showing whole text with all the \mwe{}s in single glance. Integration of the SemLex browser is also beneficial. Details of \seman\ interface are described in \Sref{sec:seman}. 

The drawbacks on the other hand are:
\bibliographystyle{plainnat}
\end{document}